\documentclass[a4paper]{article}

%\usepackage[charter]{mathdesign}
%\renewcommand*\familydefault{\sfdefault}

\usepackage{amsmath}
\usepackage{amssymb}
\usepackage{upgreek}
\usepackage{amsthm}

\usepackage{listings}

\usepackage{setspace}
\usepackage[usenames]{color}
\singlespacing
%\onehalfspacing
%\doublespacing
%\setstretch{1.1}

\parskip 2ex
\parindent 0pt

\newcommand{\present}[2]{\left \langle #1 \: | \: #2 \right \rangle} % group presentation

\newcommand{\grz}[1]{$\mathcal{E}^{#1}$}	%grzegorczyk level

\newcommand{\NN}{\mathbb{N}}	%blackboard bold
\newcommand{\ZZ}{\mathbb{Z}}

\newcommand{\least}[2]{\underset{#1}{\mu}\left( #2 \right)}	% least #1 such that #2 holds

\newcommand{\nth}{$n^{\textrm{th}}$~}	% nth and ith typeset nicely
\newcommand{\ith}{$i^{\textrm{th}}$~}

\newcommand{\shortlex}{{\sf ShortLex}\;}	% shortlex ordering has a sans-serif brand identity

\newcommand{\xvec}{\mathbf{x}}	% vector x
\newcommand{\psub}{\dot -}	% proper subtraction
\newcommand{\rsg}{\overline{sg}} % reverse signature
\newcommand{\recur}[1]{\begin{equation} \begin{split} #1 \end{split} \end{equation}}	%definition of recursive function
\newcommand{\recurN}[1]{\begin{equation*} \begin{split} #1 \end{split} \end{equation*}}	%ditto, no equation number

\theoremstyle{plain}
\newtheorem{theorem}{Theorem}[section]

\newtheorem{proposition}[theorem]{Proposition}
\newtheorem{corollary}[theorem]{Corollary}

\theoremstyle{definition}
\newtheorem{lemma}[theorem]{Lemma}
\newtheorem{definition}[theorem]{Definition}

\newenvironment{myproof}{\normalsize {\sc Proof}:}{{\hfill $\Box$}}

\newenvironment{ajd}{\noindent\color{blue} AJD }{}
\newcommand{\ad}[1]{ \begin{ajd} #1 \end{ajd}}

\newenvironment{cpe}{\noindent\color{red} CP }{}
\newcommand{\cp}[1]{ \begin{cpe} #1 \end{cpe}}


\begin{document}
\title{A stockpile of \grz{3}-computable functions}
\author{Christian Perfect}
\maketitle

\begin{abstract}
	We define the Grzegorczyk hierarchy and give explicit definitions of lots of useful functions and operations, all contained in the third level of the hierarchy.
\end{abstract}

\section{Building blocks}

\subsection{Constructing functions}
Let $\ZZ_{\geq 0}$ denote the non-negative integers. A function $f(x_1, \dots, x_n)$ is a map,
\[f: \ZZ_{\geq 0}^n \rightarrow \ZZ_{\geq 0},\]
taking $n$ parameters and resulting in a single value.

We will begin by defining the {\it primitive functions}:

\begin{itemize}
	\item {\em The zero function}. The zero function is a nullary function which returns the number $0$. For brevity I will write $0$ to mean the zero function.
	\item {\em The successor function}. The successor function $s(x) := x + 1$ is a unary function which adds $1$ to its input.
	\item {\em The projection functions}. For every $n > 0 $ and every $1 \leq i \leq n$ there is a projection function $P_n^i(x_1, \dots, x_n) := x_i$. $P_n^i$ is an $n$-ary function which returns its \ith parameter.
\end{itemize}

There are two operations we shall use to create new functions from these primitives.

\begin{itemize}
	\item {\em Composition}: $h := C(f,g_1,\dots,g_m)$ means that $h(\xvec) := f(g_1(\xvec),\dots,g_m(\xvec))$
	\item {\em Recursion}: $h := R(f,g)$ means that $h(\xvec,0) := f(\xvec)$, and $h(\xvec,n+1) := g(n,h(\xvec,n),\xvec)$
\end{itemize}

A \textit{bound variable} is one which is used to define a function, not passed as a parameter into it. For instance, in $P_n^i$, $n$ and $i$ are bound variables.

Mixing functions of different arities can be dealt with systematically. For every $n$-ary function $f$, and every $m > n$, we can define an $m$-ary function $f_m$ such that:
\[f_m(x_1, \dots, x_n, \dots, x_m) := C(f,P_m^1, \dots, P_m^n) = f(x_1, \dots, x_n).\]

Variable substitution can also be dealt with systematically:
\[f(x_1, \dots, y, \dots, x_n) := f(x_1, \dots, y(x_1, \dots, x_n), \dots, x_n) = C(f,P_n^1, \dots, y, \dots, P_n^n). \]

\subsection{Notation}
Now I will describe the notation and constructions I will use to define functions. $E \equiv E' := D$ means I will write $E$ or $E'$ to mean $D$.

\subsubsection*{Arity}
Functions in general can take any number of parameters, so it is clearly impractical to write out definitions of each operation for each possible arity. When the arity of a function $f$ doesn't matter, I will write it as $f(\xvec)$, where $\xvec$ represents an arbitrarily big vector.  When a function has to take at least a specific number of parameters $y_1, \dots, y_n$, and potentially more, I will write $f(\xvec, y_1, \dots, y_n)$

\subsubsection*{Composition} 
\[f \circ g(x) \equiv f(g(\xvec)) := C(f,g)(\xvec)\] 
or just 
\[f \circ g := C(f,g).\]

\subsubsection*{Recursion}
Definitions of recursive functions $R(f,g,h)$ will follow this format:
\begin{eqnarray*}
	f(\xvec,0) & := & g(\xvec), \\
	f(\xvec,n+1) & := & h(n,f(\xvec,n),\xvec).
\end{eqnarray*} 

Note that the parameter of recursion $n$ can be made to be in any position by composition with the projection functions:
\begin{eqnarray*}
 f'(x_1, \dots, x_k, n, x_{k+1}, \dots, x_m) &:=& f(x_1, \dots, x_m, n) \\
 &=& C(f, P_{m+1}^1, \dots, P_{m+1}^k, P_{m+1}^{m+1}, P_{m+1}^{k+1},\dots, P_{m+1}^m).
 \end{eqnarray*}

\subsubsection*{Iteration}
\[f^{(n)}(x) := \underbrace{C(f,C(f,C(f, \dots C(f,f}_{n-1 \textrm{ compositions}}) )(x).\]

\subsubsection*{Infix operators}
An infix operator is simply a function of two variables, so for any operator $\ast$ some suitable function $f_{\ast}$ could be defined:
\[x \ast y := f_{\ast}(x,y.) \]

\subsubsection*{Sets}
A set $S$ is defined by its characteristic function $\chi_S$.
\[\chi_S := \begin{cases}
 1 & x \in S, \\
 0 & x \notin S.
 \end{cases}
\]

\subsubsection*{Relations}
A relation is defined by the set of things which satisfy it. In addition, for binary relations I will write
\[ x R y := \chi_R(x,y). \]

\subsubsection*{Logic}
The concept of falsity will be represented by the number $0$, and all other values will represent truth. With this in mind, predicates $P(\xvec)$ can be written out as ordinary functions once the logical operations have been defined.

\section{The Grzegorczyk hierarchy}
The {\it Grzegorczyk hierarchy} is an infinite nested hierarchy of classes \grz{n} of functions which are closed under {\it finite composition} and {\it bounded recursion}. 

{\it Closed under finite composition}: If $f, g_1, \dots, g_m$ all belong to \grz{n}, then $C(f,g_1, \dots, g_m)$ belongs to \grz{n}. A function defined by finitely many compositions of \grz{n} functions is also an \grz{n} function.

{\it Closed under bounded recursion}: If $f,g,h$ belong to \grz{n} and $\forall \xvec$, $R(f,g)(\xvec) \leq h(\xvec)$, then $R(f,g)$ belongs to \grz{n}. In other words, if a recursion on two \grz{n} functions is bounded everywhere by another \grz{n} function, then that recursion is also in \grz{n}.

\grz{0} consists of just the primitive functions, and whatever else we can get by finitely many applications of the composition operator and bounded recursion.

\grz{n+1}, for $n \geq 0$, contains all of \grz{n} and anything obtained by a single unbounded recursion on \grz{n} functions, again closed under finite composition and bounded recursion.

The levels of the Grzegorczyk hierarchy can also be defined in terms of a stem of functions, one for each level, but we won't be going high enough for that to be useful.

\section{The stockpile of functions}
Below I will give explicit definitions for the operations of elementary arithmetic, plus some more useful things, using the notation defined above. I have tried to use definitions which, though not the most obvious, place functions as low down on the hierarchy as possible.

\section{\grz{0} functions}
Add a constant $n$ to $x$ by composing the successor function $n$ times on $x$:
\begin{equation} +_n(x) \equiv x + n := s^{(n)}(x). \end{equation}

Note that $n$ is a bound variable, that is, there is a different function $+_n$ for each $n$.

Get any constant by composing $+_n$ with $z$
\begin{equation} c_n := +_n(0). \end{equation}

From now on I will just write the number $n$ to mean the function $c_n$, when appropriate.

Decrement by 1:
\recur{
	dec(0) &:= 0, \\
	dec(n+1) &:= 1.
}

Proper subtraction:
\recur{
	x \psub 0 &:= x, \\
	x \psub (y+1) &:= dec(x \psub y).
}

$x \psub y$ is bounded by $P_2^1 \equiv x$, so belongs to \grz{0}.

Signature:
\recur{
	sg(0) &:= 0, 	\\
	sg(x+1) &:= 1.
}

And reverse signature:
\recur{
	\rsg(0) &:= 1, \\
	\rsg(x) &:= 0.
}

Equality:
\begin{equation}	x = y := eq'(x,y,sg(x \psub y)). \end{equation}
\recurN{
	eq'(x,y,0) &:= \bar{sg}(x,y), \\
	eq'(x,y,1) &:= 1.
}

$eq'$ is bounded by $c_1$, so belongs to \grz{0}.

Logical AND:
\recur{
	x \wedge 0 &:= 0,	 \\
	x \wedge (n+1) &:= x.
}

$\wedge$ is bounded by $P_2^1 \equiv x$, so is in \grz{0}.

Logical OR:
\recur{
	x \vee 0 &:= x, 		\\
	x \vee (n+1) &:= 1.
}

$\vee$ is bounded by $s \circ P_2^1 \equiv x+1$, so is in \grz{0}.

Logical NOT:
\begin{equation} \neg x := \rsg(x). \end{equation}

Ordering:
\begin{equation} \begin{split} 
	x>y &:= sg(x \psub y), 	\\
	x \geq y &:= (x > y) \vee (x=y),	\\
	x \leq y &:= \neg (x>y),	\\
	x < y &:= (x \leq y) \wedge \neg (x=y).
\end{split}\end{equation} 


The smallest of two numbers:
\begin{equation} \min(x,y) := x \psub ( x \psub y). \end{equation}

Remainder when dividing $x$ by $y$:
\recur{
	0 \bmod{y} &:= 0,	\\
	(x+1) \bmod{y} &:= rm' \left( y \psub \left( (x \bmod{y}) + 1 \right), x \bmod{y} \right).
	}
\recurN{
	rm'(0,y) &:= 0, 	\\
	rm'(x+1,y) &:= y+1.
}
$rm'$ is bounded by $y+1$ so belongs to \grz{0}. $x \bmod{y}$ is defined by recursion on \grz{0} functions and is bounded by $y-1$, so belongs to \grz{0}.

\section{\grz{1} functions}
Adding two numbers together is achieved by an unbounded recursion on the successor function.
\recur{
	x + 0 &:= x,			\\
	x + (y+1) &:= (x+y) + 1.
}
Hence addition must belong to \grz{1} and not \grz{1}.

We can define another version of $+_x$ where $x$ is unbound:
\begin{equation} +_x(y) := x + y. \end{equation}.
This version is in \grz{1}. This is really an abuse of notation, but it will be useful in a little while.

The biggest of two numbers:
\begin{equation} \max(x,y) := x + (y \psub x). \end{equation}

Note that $max(x,y)$ is bounded by one of $P_2^1$ or $P_2^2$, but I don't think there's a way of defining it without using addition, so it has to be in \grz{1}.

Absolute difference:
\begin{equation} |x - y| := (x \psub y) + (y \psub x). \end{equation}

Any constant multiple $n \cdot x$ is produced by $n$ compositions of addition:
\begin{equation} n \cdot x := +_x^{(n)}(0). \end{equation}

So we can also get any positive- and constant-coefficient linear polynomial $a_1x_1 + a_2x_2 + \dots a_nx_n$ by finite compositions. For polynomials with negative coefficients, computing all the positive terms then subtracting the negative terms will give the closest match, because we can't use negative numbers.

Quotient:
\recur{
	\left \lfloor \frac{0}{y} \right \rfloor &:= 0, \\
	\left \lfloor \frac{x+1}{y} \right \rfloor &:= \left \lfloor \frac{x}{y} \right \rfloor + \rsg\left( (x+1) \bmod{y} \right).
}

Does $x$ divide $y$:
\begin{equation} x \mid y := \rsg( x \bmod{y} ). \end{equation}

Bounded minimisation:
\recur{
	\min_{i < 0} P(\xvec,i) &:= 0,	\\
	\min_{i < y+1} P(\xvec,i) &:= \left( \min_{i  < y} P(\xvec,i) \right) + sg\left(\min_{i < y} P(\xvec,i) = y\right).
}

Because we always have to return a value, if there is no value under the bound such that the predicate holds, then we return the bound.

Bounded maximisation:
\begin{equation} \max_{i < y} P(\xvec,i) := y \psub \min_{i < y} P(\xvec, y \psub i). \end{equation}

If $P$ is a \grz{1} function then so are bounded minimisations and maximisations over $P$, because they are defined by a recursion over \grz{1} functions, bounded by $y$.

\section{\grz{2} functions}
Multiplication is an unbounded recursion on the addition operation:
\recur{
	x \cdot 0 &:= 0,				\\
	x \cdot (y+1) &:= (x \cdot y) + x.
}

We can get $x^n$, when $n$ is constant, by $n$ compositions of multiplication by $x$, and hence any (positive-coefficient) polynomial by finite compositions of addition and multiplication. So polynomials belong to \grz{2}.

Series sum:
\recur{
	\sum_{i < 0} f(\xvec, i) &:= 0,		\\
	\sum_{i < y+1} f(\xvec,i) &:= f(\xvec,y+1) + \sum_{i < y} f(\xvec,i).
}

Note that the level of the sum on the hierarchy really depends on the level of $f(\xvec,i)$. However, we can say that if $f$ is in \grz{2} then $\sum_{i < y} f(\xvec,i)$ is also in \grz{2} because it entails $y$ additions. If $y$ is a bound variable and $f$ is in \grz{1}, then we could even place the sum in \grz{1} by defining the sum as $y$ compositions of addition.

Existential quantifier:
\begin{equation} \exists_{i < y} P(\xvec,i) := \sum_{i < y} P(\xvec,i). \end{equation}


Definition by cases: if  $P_i(\xvec)$, $1 \leq i \leq k$ is a finite collection of mutually exclusive, exhaustive predicates, then we can define
\[ f(x) := \begin{cases}
	g_1(\xvec) & \textrm{if }P_1(\xvec),	\\
	\vdots & \vdots	\\
	g_k(\xvec) & \textrm{if } P_k(\xvec).
\end{cases} \]

In our language, this is:
\begin{equation} f(x) := \sum_{i < k+1} g_i(x) \cdot sg(P_i(x)), \end{equation}
where $k$ is a bound variable. If all the $P_i$ and $g_i$ are in \grz{2}, then so is $f$, because we are just doing $k$ multiplications and additions.

\section{\grz{3} functions}
\grz{3} is where just about all the really useful operations are.

First of all, we can get exponentiation by unbounded recursion on the multiplication operation:
\recur{
	x^0 &:= 1,				\\
	x^{y+1} &:= x \cdot x^y.
}

Factorial:
\recur{
	0! &:= 1,	\\
	(x+1)! &:= (x+1) \cdot x!.
}

Series product:
\recur{
	\prod_{i < 0} f(\xvec,i) &:= 1,	\\
	\prod_{i < y+1} f(\xvec,i) &:= f(\xvec,y+1) \cdot \prod_{i < y} f(\xvec, i).
}
If $f$ is in \grz{3} then so is the product $\prod_{i < y} f(\xvec,i)$, as it consists of $y$ multiplications. If $y$ is a bound variable, then we can define the product using $y$ compositions of multiplication instead of the recursion, so if $f$ is in \grz{2} then the series product is too.

Universal quantifier:
\begin{equation} \forall_{i < y} P(\xvec,i) := \prod_{i < y} P(\xvec,i). \end{equation}

Primality:
\begin{equation} x \textrm{ prime} :=  (x > 1) \wedge \forall_{z < x} \left( (z < 2) \vee \neg (z \mid x) \right). \end{equation}

The first prime after $x$:
\begin{equation} \operatorname{nextprime}(x) := \min_{i \leq x!+1} (i \textrm{ prime}) \wedge (i > x). \end{equation}
The bound on this function comes from Euclid or some such bearded Greek. Note that this is an exceptionally inefficient way of finding primes.

The \nth prime:
\recur{
	p_0 \equiv p(0) &:= 2,		\\
	p_{n+1} \equiv p(n+1) &:= \operatorname{nextprime} (p_n).
}

Exponent of $x$ in the decomposition of $y$:
\begin{equation} [y]_x := \max_{i < y} \left( x^i \mid y \right). \end{equation}

\subsection{Lists}

G\"odel encoding of an $n$-tuple:
\begin{equation} \xvec \equiv [x_1, \dots, x_n] := \prod_{1 \leq i \leq n} p_i^{x_i+1}. \end{equation}
The $+1$ in the exponent is so we can reliably determine the length of a list. The bold $\xvec$ is shorthand so I don't have to write out the brackets formation. It shouldn't be confused with the use of $\xvec$ previously to mean an arbitrary number of parameters. An encoded list $\xvec$ is just a single number.

Value of \nth position in an encoded list:
\begin{equation} (\xvec)_i := [\xvec]_{p_i} - 1. \end{equation}

Length of an encoded list:
\begin{equation} |\xvec| := \min_{i < \xvec} \left([\xvec]_{p_i} = 0  \right).\end{equation}

The index of the first occurrence of a given value in an encoded list:
\begin{equation} \operatorname{find}(\xvec,n) := \min_{i < |\xvec|} (\xvec)_i = y. \end{equation}

Whether an encoded list contains a given value:
\begin{equation} y \in \xvec := \operatorname{find}(\xvec,y) < |\xvec|. \end{equation}

The largest element of an encoded list:
\begin{equation} \operatorname{biggest}(\xvec) := \operatorname{biggest}'(\xvec,|\xvec|) \end{equation}
\recurN{
	\operatorname{biggest}'(\xvec,0) &:= (\xvec)_0 \\ 
	\operatorname{biggest}'(\xvec,n+1) &:= \max \left ( (\xvec)_{n+1},\operatorname{biggest}'(\xvec,n) \right )
}

To obtain the sub-list of $\xvec$ starting at position $s$ and ending at position $e$.:
\begin{equation} \xvec[s \dots e] \equiv \operatorname{slice}(\xvec,s,e) := \prod_{i=s'}^{e'}(p_{i+1-s'})^{(\xvec)_i+1} \end{equation}
\begin{equation*} 
\begin{split}
	e' &:= \max(1,\min(e,|\xvec|)) \\ 
	s' &:= \max(1,\min(s,e')) 
\end{split}
\end{equation*}

To append a single value to the end of an encoded list:
\begin{equation} \operatorname{splice}(\xvec,n) := \xvec \cdot p_{|\xvec|}^{n+1}. \end{equation}

\end{document}

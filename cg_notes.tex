\documentclass[a4paper]{article}
\usepackage{amsmath}
\usepackage{amssymb}
\usepackage{upgreek}
\usepackage{amsthm}
\usepackage{verbatim}
\usepackage{enumerate}
\usepackage[usenames,dvipsnames]{color}
\usepackage{setspace}
\usepackage{xr}
\singlespacing
%\onehalfspacing
%\doublespacing
%\setstretch{1.1}

\parskip 2ex
\parindent 0pt

\newcommand{\grz}[1]{$\mathcal{E}^{#1}$}	%grzegorczyk level

\newcommand{\NN}{\mathbb{N}}	%blackboard bold
\newcommand{\ZZ}{\mathbb{Z}}
\newcommand\ov[1]{\overline{#1}} %overline
\newcommand{\maps}{\longrightarrow}

\newcommand\eps{\varepsilon}

\newcommand{\nth}{$n^{\textrm{th}}$~}	% nth and ith typeset nicely
\newcommand{\ith}{$i^{\textrm{th}}$~}
\renewcommand{\i}{\iota}
\renewcommand{\t}{\tau}

\newcommand{\shortlex}{{\sf ShortLex}\;}	% shortlex ordering has a sans-serif brand identity

\newcommand{\avec}{\mathbf{a}}	% vector
\newcommand{\bvec}{\mathbf{b}}	% vector 
\newcommand{\cvec}{\mathbf{c}}	% vector 
\newcommand{\dvec}{\mathbf{d}}	% vector 
\newcommand{\uvec}{\mathbf{u}}	% vector 
\newcommand{\vvec}{\mathbf{v}}	% vector 
\newcommand{\wvec}{\mathbf{w}}	% vector w
\newcommand{\xvec}{\mathbf{x}}	% vector x
\newcommand{\yvec}{\mathbf{y}}	% vector y
\newcommand{\zvec}{\mathbf{z}}	% vector 
\newcommand{\tvec}{\mathbf{t}}	% vector t
\newcommand{\Uvec}{\mathbf{U}}	% vector U
\newcommand{\psub}{\dot -}	% proper subtraction
\newcommand{\rsg}{\overline{sg}} % reverse signature
\newcommand{\recur}[1]{\begin{equation} \begin{split} #1 \end{split} \end{equation}}	%definition of recursive function
\newcommand{\recurN}[1]{\begin{equation*} \begin{split} #1 \end{split} \end{equation*}}	%ditto, no equation number

\newcommand{\classC}{$\mathcal{C}$}

\newcommand{\concat}{\ensuremath{+\!\!\!\!+\,}}	% list concatenation

\newcommand{\present}[2]{\left \langle #1 \: | \: #2 \right \rangle}	%group presentation
\newcommand{\fgoagog}{\pi_1(\mathbf{G},\Gamma,T,v_0)}	%fundamental group of a graph of groups
%universal group of a pregroup P
\newcommand{\UP}{\Uvec(P)}

%%% sets { ... | ... }
\newcommand{\set}[2]{\left\{\, \mathinner{#1}\vphantom{#2}\; \left|\; \vphantom{#1}\mathinner{#2} \right.\,\right\}}
\newcommand{\oneset}[1]{\left\{\, \mathinner{#1} \,\right\}}
\newcommand{\smallset}[1]{\left\{\mathinner{#1}\right\}}
%%%%%%%%%%%% brackets etc
\newcommand{\abs}[1]{\left|\mathinner{#1}\right|}
\newcommand{\floor}[1]{\left\lfloor\mathinner{#1} \right\rfloor}
\newcommand{\ceil}[1]{\left\lceil\mathinner{#1} \right\rceil}
\newcommand{\bracket}[1]{\left[\mathinner{#1} \right]}
\newcommand{\parenth}[1]{\left(\mathinner{#1} \right)}
\newcommand{\gen}[1]{\left< \mathinner{#1} \right>}

\newcommand{\rdeg}[1]{\mbox{red-deg}\left(\mathinner{#1}\right)}
%%%%%%%%%%%%%functions
\newcommand{\find}{\operatorname{find}}
\newcommand{\Dfind}{\operatorname{Dfind}}
\newcommand{\Preduced}{\operatorname{Preduced}}
\newcommand{\leftm}{\operatorname{leftm}}
\newcommand{\Predn}{\operatorname{Predn}}
\newcommand{\Preduction}{\operatorname{Preduction}}
\newcommand{\Interleaven}{\operatorname{Interleaven}}
\newcommand{\GN}{\operatorname{GN}}
\newcommand{\+}{\:\hat +\:}
\newcommand{\�}{\hat -}
\newcommand{\biggest}{\operatorname{biggest}}
\newcommand{\bigL}{\operatorname{bigL}}
\newcommand{\bigR}{\operatorname{bigR}}

\newcommand{\be}{\begin{enumerate}}
\newcommand{\ee}{\end{enumerate}}

\theoremstyle{plain}
\newtheorem{theorem}{Theorem}[section]

\newtheorem{proposition}[theorem]{Proposition}
\newtheorem{corollary}[theorem]{Corollary}

\theoremstyle{definition}
\newtheorem{lemma}[theorem]{Lemma}
\newtheorem{definition}[theorem]{Definition}

\newenvironment{myproof}{\normalsize {\sc Proof}:}{{\hfill $\Box$}}

\newenvironment{cpe}{\noindent\color{OliveGreen} CP }{}
\newcommand{\cp}[1]{
\begin{cpe} #1 \end{cpe}}
%%
\newenvironment{ad}{\noindent\color{blue} AJD }{}
\newcommand{\ajd}[1]{
\begin{ad} #1 \end{ad}}
%%
\externaldocument{bass_serre_computability}
\begin{document}
\title{Notes on Cannonito and Gatterdam, Computability of group const.}
\author{Christian Perfect \and Andrew Duncan}
\maketitle

\section*{Theorem 3.1}
We shall define Cannonito and Gatterdam's 
 standard index of $F$ using the functions defined
in the Bass-Serre computability paper, as well as those defined, below. 
All references to equation numbers $<100$ are to the Basse-Serre paper.
\addtocounter{equation}{100} 

$J(x,y)$ is the  integer pairing function $J:\ZZ_{\geq 0} \times \ZZ \rightarrow \ZZ_{\geq 0}$ defined by 
\begin{equation}
J(x,y):=((x+y)^2+x)^2+y.
\end{equation}
This has ``left inverse'' $L$ and ``right inverse'' $R$ (called $K$ and
$L$ in \cite{Cannonito_1973}) given by
\begin{align}
R(x)&:=x\psub \lfloor x^{1/2}\rfloor^2 \textrm{ and}\\
L(x)&:=R(\lfloor x^{1/2}\rfloor)=\lfloor x^{1/2}\rfloor \psub \lfloor\lfloor x^{1/2} \rfloor^{1/2} \rfloor^2,
\end{align}
(where $\lfloor x^{1/2}\rfloor$ is the greatest integer $y$ such that 
$y^2<x$, and is defined in the Bass-Serre paper). 
The functions $J$, $L$ and $R$ satisfy 
\[LJ(x,y)=x\textrm{ and } RJ(x,y)=y\] 
and from \cite{Ritchie_1965} it follows that they are \grz{3}-computable.

For an integer $n$ define
\begin{equation} 
\bar n: = 2n, \textrm{ if } n\ge 0,\textrm{ and } -2n-1 \textrm{ otherwise}.
\end{equation}
To define the multiplication and inverse functions for  the free 
group it will be useful to have functions $\+$ and $\�$ such that, 
$\overline{m+n}=\bar m \+ \bar n$ and $\overline{-n}=\� \bar n$.
These are defined, for $x,y\in \ZZ_{\geq 0}$,  by 
\begin{equation}
x\+ y :=
\begin{cases}
x+y &\textrm{ if } x \bmod 2=y \bmod 2=0 \\ 
x-y-1 &\textrm{ if } x \bmod 2=0, y \bmod 2=1 \textrm{ and } x\ge y+1\\ 
-x+y &\textrm{ if }  x \bmod 2=0, y \bmod 2=1 \textrm{ and } x< y+1\\ 
-x+y-1 &\textrm{ if } x \bmod 2=1, y \bmod 2=0 \textrm{ and } y\ge x+1\\ 
x-y &\textrm{ if } x \bmod 2=1, y \bmod 2=0 \textrm{ and } y< x+1 \\ 
x+y+1 &\textrm{ if } x \bmod 2=y \bmod 2=1
\end{cases}
\end{equation}
and 
\begin{equation}
\� x :=
\begin{cases}
0&\textrm{ if } x=0\\
x-1&\textrm{ if } x\neq 0\textrm{ and } x \bmod 2=0\\
x+1 &\textrm{ if } x\neq 0\textrm{ and } x \bmod 2=1
\end{cases}
.
\end{equation}
These are both \grz{3}-computable functions and have the required
properties. In particular $\bar m\+\bar n=0$ if and only if $m+n=0$.  

Let $F$ be free on a finite or  countable set $\{a_1,\ldots \}$. 
We identify $F$ with the set of reduced words on its generating set. 
Define
an index $i:F\maps \ZZ_{\geq 0}$ by 
\begin{align}
i(\emptyset)&:= 1\notag\\
i(a_{i_0}^{n_0}\cdots a_{i_r}^{n_r})&:=[J(i_0,\bar n_0),\ldots, J(i_r,\bar n_r)],\label{al:ifree}
\end{align}
for $r\ge 0$, $n_k\in \ZZ$, $n_k\neq 0$, 
and $a_{i_k}\neq a_{i_{k+1}}$, where $[\xvec]$ is defined in 
\eqref{eq:ntuple}. This is slightly different from the index defined in
\cite{Cannonito_1973}, where $J(i_k,\bar n_k)-1$ is used instead of
$J(i_k,\bar n_k)$; the difference arising because of the $+1$ in
\eqref{eq:ntuple}. 

Define the predicate $\GN$, which checks for the G\"odel number of 
a sequence of positive integers, by
\begin{equation}
\GN(\xvec) \iff 0\notin \xvec.
\end{equation} 
If $F$ is countably generated, by $\{a_1,\ldots \}$, then 
\[\xvec\in i(F)\iff \xvec=1 \vee [\GN(\xvec) \wedge \min_k\{R((\xvec)_k)=0\}=|\xvec|]. \]
If $F$ is generated by $\{a_1,\ldots, a_{n-1}\}$, we have 
\[\xvec\in i(F)\iff \xvec=1 \vee [\GN(\xvec) \wedge \min_k\{[L((\xvec)_k)>n-1]\vee [R((\xvec)_k)=0]\}=|\xvec|]. \]
Therefore $i$ is \grz{3}-decidable. 


To make the definition of $m$ easier to read, for an encoded list
$\xvec$ write $\xvec_\t$ for $L((\xvec)_{|\xvec|-1})$, $\xvec_\i$ for 
$L((\xvec_0))$, $n(\xvec)_\t$ for $R((\xvec)_{|\xvec|-1})$ and 
$n(\xvec)_\i$ for $R((\xvec_0))$ (so if $\xvec=i(a_{i_0}^{n_0}\cdots a_{i_r}^{n_r})$
then $\xvec_\i=i_0$, $\xvec_\t=i_r$, $n(\xvec)_\t=\bar n_r$ and $n(\xvec)_\i =\bar n_0$.)
Define
\begin{equation}\label{eq:mfree}
m(\xvec,\yvec) :=
\begin{cases}
 \xvec \concat \yvec& \textrm{ if } \xvec_\t\neq \yvec_\i\\
\xvec[0..|\xvec|-2]&\\
\quad\concat [J(\xvec_\t,\overline{n(\xvec)_\t}\+
\overline{n(\yvec)_\i})]&\\
\quad\quad\concat\yvec[1..|\yvec|-1]& \textrm{ if } \xvec_\t = \yvec_\i \wedge 
(\overline{n(\xvec)_\t}\+
\overline{n(\yvec)_\i}\neq 0)\\
m(\xvec[0..|\xvec|-2],\yvec[1..|\yvec|-1])& \textrm{ if }\xvec_\t = \yvec_\i \wedge 
(\overline{n(\xvec)_\t}\+
\overline{n(\yvec)_\i} =0)
\end{cases}
\end{equation}
and 

\begin{equation}\label{eq:jfree}
j(\xvec) :=j_1(\xvec,|\xvec|),
\end{equation} 
where 
\begin{align*}
j_1(\xvec,0) &:= 1 \textrm{ and }\\
j_1(\xvec, n+1) &:= 
\begin{cases}[J(L((\xvec)_n),\� R((\xvec)_n))]\concat j_1(\xvec,n)
& \textrm{ if } n+1\le |\xvec|\\
j_1(\xvec, n)& \textrm{ otherwise }
\end{cases}
\end{align*}
If $\xvec=i(u)$ and $\yvec=i(v)$, where $u$ and $v$ are reduced words 
in $F$ then $m(\xvec,\yvec)=i(uv)$ and $j(\xvec)=i(u^{-1})$, as required.
Also, $m$ and $j$ are \grz{3}-computable, as observed in \cite{Cannonito_1973},
since $m(\xvec,\yvec)\le \xvec \concat \yvec$ and \[j_1(\xvec)\le 
p_{|\xvec|}^{|\xvec|(J(\bigL,\bigR + 1)+1)},\]
where 
\[\bigL:=\biggest[L((\xvec)_0),\ldots, L((\xvec)_{|\xvec|-1})]\] and 
\[\bigR:=\biggest[R((\xvec)_0),\ldots, R((\xvec)_{|\xvec|-1})].\]

\begin{definition}
Let $F$ be free on a finite or  countable set $\{a_1,\ldots \}$. 
Then $F$ is \grz{3}-computable with respect to the index $(i,m,j)$ defined
in \eqref{al:ifree}, \eqref{eq:mfree} and \eqref{eq:jfree} above. This index is called the 
\emph{standard index} of $F$ (w.r.t $\{a_1,\ldots \}$).
\end{definition}
\section{Definition 3.3}
Let $G$ be a group with presentation $\langle X|R\rangle$, let $F=F(X)$ be the 
free group on $X$ and let $K$ be the normal closure of $R$ in $F(X)$. 
Assume that $X$ is countable or finite and   
let $(i,m,j)$ be the standard index of $F$ w.r.t. $X$. 
Define the predicate $E$ by
\begin{equation}
E(x,y)\iff x\in i(F)\wedge y\in i(F) \wedge m(j(x),y))\in i(K).
\end{equation}
Define the function $r$ by
\begin{equation}
r(x):=\min_y(y\le x \wedge E(x,y)).
\end{equation}
\begin{definition}
The \emph{standard index} of $G$, with respect to the presentation
$P:=\langle X| R\rangle$, is $(i_P,m_P,j_P)$, where
\begin{equation}
i_P:=ri,\: m_P:=rm\textrm{ and } j_P:=rj.
\end{equation}
The group $G$ is said to be \emph{\grz{n}(A) standard}, if there 
is a presentation $P$ for $G$ such that $(i_P,m_P,j_P)$ is a \grz{n}(A) computable index for $G$.
\end{definition}
\section{Theorem 3.4}
\begin{theorem}[\cite{Cannonito_1973}, Theorem 3.4]
Let $G$ be a finitely generated  \grz{n}(A) standard group and let
$Q:=\langle Y|S\rangle$ be a presentation for $G$, where $Y$ is finite. 
Then the standard index $(i_Q,m_Q,j_Q)$ for $G$ is \grz{n}(A) computable. 
\end{theorem}
\begin{proof}
As $G$ is a \grz{n}(A) standard group there is a presentation 
$P=\langle X|R\rangle$ for $G$ such that the index $(i_P,m_P,j_P)$ is
\grz{n}(A) computable. 

Let $(i,m,j)$ and $(i^\prime, m^\prime, j^\prime)$ be the standard indices
of $F(X)$ and $F(Y)$, respectively. 
Let $\pi_P$%:F(X)\maps G$ 
 and $\pi_Q$ be the canonical maps from $F(X)$ to
$G$ and from $F(Y)$ to $G$. As noted in \cite{Cannonito_1973} it suffices
to show that the map $\pi_Q$ is \grz{n}(A) computable with respect to
the indices $(i^\prime,m^\prime,j^\prime)$ and $(i_P,m_P,j_P)$. That is,
to show that the map $\hat\pi_Q:i^\prime(F(Y))\maps i_P(G)$, 
given by $\hat\pi_Q(i^\prime(w))=i_P(\pi_Q(w))$, where  $w\in F(Y)$, is 
\grz{n}(A) computable. 

Let $Y=\{y_1,\ldots, y_n\}$. The maps $\pi_P$ and $\pi_Q$ are surjective, so
for each $k$ we may choose $v_k\in F(X)$ such that $\pi_Q(y_k)=\pi_P(v_k)$. 
Let $M=\max\{i(v_k^\eps): k=1,\ldots n, \eps=\pm 1\}$. 
Now let $w=y_{k_0}^{\eps_0}\cdots y_{k_r}^{\eps_r}\in F(Y)$, with $y_{k_s}
\in Y$ and $\eps_s=\pm 1$. Then
\begin{align*}
\hat\pi_Q(i^\prime(w)) & = \i_P(\pi_Q(w))\\
&=i_P(\pi_Q(y_{k_0}^{\eps_0})\cdots\pi_Q(y_{k_r}^{\eps_r}))\\
&=i_P(\pi_P(v_{k_0}^{\eps_0})\cdots  \pi_P(v_{k_r}^{\eps_r})).%\\
%&=m_P(\ldots  m_P(\pi_P(v_{k_0}^{\eps_0}), \pi_P(v_{k_1}^{\eps_1})) \ldots
%, \pi_P(v_{k_r}^{\eps_r})).
\end{align*} 
%(That is 
The index of the product on the 3rd line is equal to the 
number obtained by $r$ applications of $m_P$, beginning with the 
leftmost pair. More precisely, define
\begin{align*}
\hat m_P(\xvec,0)&:=1 \textrm{ and }\\
\hat m_P(\xvec, n+1)&:= 
\begin{cases}
m_P(\hat m_P(\xvec,n),(\xvec)_{n+1})
& \textrm{ if } n+1< |\xvec|\\
\hat m_P(\xvec,n)& \textrm{ otherwise }
\end{cases}
\end{align*}
Then 
\[i_P(\pi_P(v_{k_0}^{\eps_0})\cdots  \pi_P(v_{k_r}^{\eps_r}))
= \hat m_P([i_P(\pi_P(v_{k_0}^{\eps_0})), \ldots ,
\i_P(\pi_P(v_{k_r}^{\eps_r}) ]).\]

As $(i_P,m_P,j_P)$ is a standard index we have
$\hat m_P(\xvec)\le M\concat \cdots \concat M$, ($r$ $\concat$'s)
so $\hat m_P\le p_{|w|}^{|w|(M+1)}$, and it follows that $\hat \pi_Q$ is
\grz{n}(A) computable.
\end{proof}

Note that in the statement of the theorem above it is not assumed that
the original standard index is defined with respect to a presentation
with a finite generating set. However, if the original generating set,
$X$ in the proof, is countably infinite, then
  since $G$ is finitely generated, we may may choose generating elements
$g_1,\ldots ,g_n$, and then $g_k$ is a word in finitely many elements of $X$,
for $k=1,\ldots ,n$. The union $X^\prime$ of these elements is enough to generate
$G$, and every word in $R$ can then be rewritten as a word over $X^\prime$, to
give a new presentation $\langle X^\prime|R^\prime\rangle$, with 
$X^\prime$ finite.
The standard index with respect to this presentation is now \grz{n}(A)
computable, by the Theorem. Thus, once we have the theorem, we may
 assume that if $G$ is a f.g. \grz{n}(A) standard group then it
has a \grz{n}(A) standard index with respect to  a presentation on
a finite generating set.   
\bibliographystyle{alpha}
\bibliography{grzegorczyk}

\end{document}

\documentclass[a4paper]{article}
\usepackage{amsmath}
\usepackage{amssymb}
\usepackage{upgreek}
\usepackage{amsthm}
\usepackage{verbatim}
\usepackage{enumerate}
\usepackage[usenames,dvipsnames]{color}
\usepackage{setspace}
\usepackage{xr}
\singlespacing
%\onehalfspacing
%\doublespacing
%\setstretch{1.1}

\parskip 2ex
\parindent 0pt

\newcommand{\grz}[1]{$\mathcal{E}^{#1}$}	%grzegorczyk level

\newcommand{\NN}{\mathbb{N}}	%blackboard bold
\newcommand{\ZZ}{\mathbb{Z}}
\newcommand\ov[1]{\overline{#1}} %overline
\newcommand{\maps}{\longrightarrow}

\newcommand\eps{\varepsilon}

\newcommand{\nth}{$n^{\textrm{th}}$~}	% nth and ith typeset nicely
\newcommand{\ith}{$i^{\textrm{th}}$~}

\newcommand{\shortlex}{{\sf ShortLex}\;}	% shortlex ordering has a sans-serif brand identity

\newcommand{\avec}{\mathbf{a}}	% vector
\newcommand{\bvec}{\mathbf{b}}	% vector 
\newcommand{\cvec}{\mathbf{c}}	% vector 
\newcommand{\dvec}{\mathbf{d}}	% vector 
\newcommand{\uvec}{\mathbf{u}}	% vector 
\newcommand{\vvec}{\mathbf{v}}	% vector 
\newcommand{\wvec}{\mathbf{w}}	% vector w
\newcommand{\xvec}{\mathbf{x}}	% vector x
\newcommand{\yvec}{\mathbf{y}}	% vector y
\newcommand{\zvec}{\mathbf{z}}	% vector 
\newcommand{\tvec}{\mathbf{t}}	% vector t
\newcommand{\Uvec}{\mathbf{U}}	% vector U
\newcommand{\psub}{\dot -}	% proper subtraction
\newcommand{\rsg}{\overline{sg}} % reverse signature
\newcommand{\recur}[1]{\begin{equation} \begin{split} #1 \end{split} \end{equation}}	%definition of recursive function
\newcommand{\recurN}[1]{\begin{equation*} \begin{split} #1 \end{split} \end{equation*}}	%ditto, no equation number

\newcommand{\classC}{$\mathcal{C}$}

\newcommand{\concat}{\ensuremath{+\!\!\!\!+\,}}	% list concatenation

\newcommand{\present}[2]{\left \langle #1 \: | \: #2 \right \rangle}	%group presentation
\newcommand{\fgoagog}{\pi_1(\mathbf{G},\Gamma,T,v_0)}	%fundamental group of a graph of groups
%universal group of a pregroup P
\newcommand{\UP}{\Uvec(P)}

%%% sets { ... | ... }
\newcommand{\set}[2]{\left\{\, \mathinner{#1}\vphantom{#2}\; \left|\; \vphantom{#1}\mathinner{#2} \right.\,\right\}}
\newcommand{\oneset}[1]{\left\{\, \mathinner{#1} \,\right\}}
\newcommand{\smallset}[1]{\left\{\mathinner{#1}\right\}}
%%%%%%%%%%%% brackets etc
\newcommand{\abs}[1]{\left|\mathinner{#1}\right|}
\newcommand{\floor}[1]{\left\lfloor\mathinner{#1} \right\rfloor}
\newcommand{\ceil}[1]{\left\lceil\mathinner{#1} \right\rceil}
\newcommand{\bracket}[1]{\left[\mathinner{#1} \right]}
\newcommand{\parenth}[1]{\left(\mathinner{#1} \right)}
\newcommand{\gen}[1]{\left< \mathinner{#1} \right>}

\newcommand{\rdeg}[1]{\mbox{red-deg}\left(\mathinner{#1}\right)}
%%%%%%%%%%%%%functions
\newcommand{\find}{\operatorname{find}}
\newcommand{\Dfind}{\operatorname{Dfind}}
\newcommand{\Preduced}{\operatorname{Preduced}}
\newcommand{\leftm}{\operatorname{leftm}}
\newcommand{\Predn}{\operatorname{Predn}}
\newcommand{\Preduction}{\operatorname{Preduction}}
\newcommand{\Interleaven}{\operatorname{Interleaven}}

\newcommand{\be}{\begin{enumerate}}
\newcommand{\ee}{\end{enumerate}}

\theoremstyle{plain}
\newtheorem{theorem}{Theorem}[section]

\newtheorem{proposition}[theorem]{Proposition}
\newtheorem{corollary}[theorem]{Corollary}

\theoremstyle{definition}
\newtheorem{lemma}[theorem]{Lemma}
\newtheorem{definition}[theorem]{Definition}

\newenvironment{myproof}{\normalsize {\sc Proof}:}{{\hfill $\Box$}}

\newenvironment{cpe}{\noindent\color{OliveGreen} CP }{}
\newcommand{\cp}[1]{
\begin{cpe} #1 \end{cpe}}
%%
\newenvironment{ad}{\noindent\color{blue} AJD }{}
\newcommand{\ajd}[1]{
\begin{ad} #1 \end{ad}}
%%
\externaldocument{bass_serre_computability}
\begin{document}
\title{Notes on Cannonito and Gatterdam, Computablity of group const.}
\author{Christian Perfect \and Andrew Duncan}
\maketitle

\section*{Theorem 3.1}


$J(x,y)$ is the  integer pairing function $J:\ZZ_{\geq 0} \times \ZZ \rightarrow \ZZ_{\geq 0}$ defined by 
\[J(x,y)=((x+y)^2+x)^2+y.\]
This has ``left inverse'' $L$ and ``right inverse'' $R$ (called $K$ and
$L$ in \cite{Cannonito_1973}) given by
\begin{align*}
R(x)&=x\psub \lfloor x^{1/2}\rfloor^2 \textrm{ and}\\
L(x)&=R(\lfloor x^{1/2}\rfloor)=\lfloor x^{1/2}\rfloor \psub \lfloor\lfloor x^{1/2} \rfloor^{1/2} \rfloor^2,
\end{align*}
where $\lfloor x^{1/2}\rfloor$ is the greatest integer $y$ such that 
$y^2<x$. 
The functions $J$, $L$ and $R$ satisfy 
\[LJ(x,y)=x\textrm{ and } RJ(x,y)=y\] 
and from \cite{Ritchie_1965} it follows that they are \grz{3}-computable.

For an integer $n$ define 
\[\bar n= 2n, \textrm{ if } n\ge 0,\textrm{ and } -2n-1 \textrm{ otherwise}.\]

Let $F$ be free on a finite or  countable set $\{a_1,\ldots \}$. 
We shall define Cannonito and Gatterdam's 
 standard index of $F$ using the functions defined
in the Bass-Serre computability paper, as well as $J$, $L$ and $R$. 
All references to equation numbers are to the Basse-Serre paper. 
We identify $F$ with the set of reduced words on its generating set. 
Define
an index $i:F\maps \ZZ_{\geq 0}$ by 
\begin{align*}
i(\emptyset)&= 1\\
i(a_{i_0}^{n_0}\cdots a_{i_r}^{n_r})&=[J(i_0,\bar n_0)-1,\ldots, J(i_r,\bar n_r)-1],
\end{align*}
for $r\ge 0$, $n_k\in \ZZ$, $n_k\neq 0$, 
and $a_{i_k}\neq a_{i_{k+1}}$, where $[x]$ is defined in \eqref{eq:ntuple}. 

\bibliographystyle{alpha}
\bibliography{grzegorczyk}

\end{document}

\documentclass[12pt,a4paper,draft]{article}
\usepackage{times,amsmath,amsthm,amssymb,tabularx,hhline,rotating,enumerate,comment}
\usepackage{amsmath,amsthm,amsfonts,amssymb}
\usepackage{graphicx}
\usepackage{hyperref}
\usepackage[usenames]{color}
\pagestyle{myheadings} \markright{{\sl  Pregroups on the Grzegorczyk hierarchy}}


\newcommand{\SIG}{\Sigma}
\newcommand{\GG}{\Gamma}
\newcommand{\gp}{geodesically perfect}

\newcommand{\IFF}{if and only if } % \iff is already defined
\newcommand{\set}[2]{\left\{\, \mathinner{#1}\vphantom{#2}\; \left|\; \vphantom{#1}\mathinner{#2} \right.\,\right\}}
\newcommand{\oneset}[1]{\left\{\, \mathinner{#1} \,\right\}}
\newcommand{\smallset}[1]{\left\{\mathinner{#1}\right\}}
\newcommand{\abs}[1]{\left|\mathinner{#1}\right|}
\newcommand{\floor}[1]{\left\lfloor\mathinner{#1} \right\rfloor}
\newcommand{\ceil}[1]{\left\lceil\mathinner{#1} \right\rceil}
\newcommand{\bracket}[1]{\left[\mathinner{#1} \right]}
\newcommand{\parenth}[1]{\left(\mathinner{#1} \right)}
\newcommand{\gen}[1]{\left< \mathinner{#1} \right>}

\newcommand{\rdeg}[1]{\mbox{red-deg}\left(\mathinner{#1}\right)}
\newcommand\IRR{\mathop\mathrm{IRR}}
\newcommand\eps{\varepsilon}

\newcommand\ov[1]{\overline{#1}}
\newcommand\ovv[2]{\overline{#1}\,\overline{#2}}

\newcommand{\links}{[\![}
\newcommand{\rechts}{]\!]}

\newcommand\cE{\mathcal{E}}
\newcommand\cG{\mathcal{G}}
\newcommand\cB{\mathcal{B}}
\newcommand\cH{\mathcal{H}}
\newcommand\cM{\mathcal{M}}
\newcommand\cU{\mathcal{U}}
\newcommand{\D}{\Delta}
\newcommand{\e}{\epsilon}
\renewcommand{\l}{\lambda}
\newcommand{\Oh}{\mathcal{O}}
\newcommand{\oh}{o} % {\mathcal{o}}
\newcommand\ZZ{\mathbb{Z}}
\newcommand\NN{\mathbb{N}}
\newcommand\RR{\mathbb{R}}

\newcommand{\aside}[1]{\marginpar{\footnotesize #1}}

\newcommand\HNN{\mathrm{HNN}}

\newcommand{\maps}{\longrightarrow}
\newcommand\RAS[2]{\overset{#1}{\underset{#2}{\Longrightarrow}}}
\newcommand\ras[2]{\overset{#1}{\underset{#2}{\longrightarrow}}}
\newcommand\LAS[2]{\overset{#1}{\underset{#2}{\Longleftarrow}}}
\newcommand\las[2]{\overset{#1}{\underset{#2}{\longleftarrow}}}
\newcommand\DAS[2]{\overset{#1}{\underset{#2}{\Longleftrightarrow}}}
\newcommand\das[2]{\overset{#1}{\underset{#2}{\longleftrightarrow}}}
\newcommand\OUTS[5]{#1
\overset{#2}{\underset{#3}{\Longleftarrow}} #4
\overset{#2}{\underset{#3}{\Longrightarrow}} #5}
\newcommand\INS[5]{#1
\overset{#2}{\underset{#3}{\Longrightarrow}} #4
\overset{#2}{\underset{#3}{\Longleftarrow}} #5}
\newcommand\RA[1]{{\underset{#1}{\Longrightarrow}}}
\newcommand\LA[1]{{\underset{#1}{\Longleftarrow}}}
\newcommand\DA[1]{{\underset{#1}{\Longleftrightarrow}}}
\newcommand\OUT[4]{#1
{\underset{#2}{\Longleftarrow}} #3
{\underset{#2}{\Longrightarrow}} #4}
\newcommand\IN[4]{#1
{\underset{#2}{\Longrightarrow}} #3
{\underset{#2}{\Longleftarrow}} #4}


\newtheorem{theorem}{{\bf Theorem}}[section]
\newtheorem{corollary}[theorem]{{\bf Corollary}}
\newtheorem{definition}[theorem]{{\bf Definition}}
\newtheorem{example}[theorem]{{\bf Example}}
\newtheorem{lemma}[theorem]{{\bf Lemma}}
\newtheorem{proposition}[theorem]{{\bf Proposition}}
\newtheorem{remark}[theorem]{{\bf Remark}}
\newtheorem{problem}[theorem]{{\bf Problem}}
\newtheorem{conjecture}[theorem]{{\bf Conjecture}}


\newenvironment{am}{\noindent\color{blue} AM: }{}
\newenvironment{ajd}{\noindent\color{red} AJD }{}
\newenvironment{vd}{\noindent\color{green} VD }{}
\newcommand{\vvd}[1]{
\begin{vd} #1 \end{vd}}

\newcommand{\aam}[1]{ \begin{am} #1 \end{am}}
\newcommand{\ad}[1]{ \begin{ajd} #1 \end{ajd}}



\newcommand{\be}{\begin{enumerate}}
\newcommand{\ee}{\end{enumerate}}
\begin{document}

\title{The Grzegorczyk hierarchy and pregroups}
\author{}

\date{\today}

\maketitle

\begin{abstract}
The level of the universal group of a pregroup is at most one
more than the level of the pregroup; under suitable conditions. 
\end{abstract}

\section{Rabin}\label{sec:rabin}
For $\cE_n$ read $\cE_n(A)$.\\[1em]

It may be best to define an index as a triple $(i,j,k)$ as in Cannonito \& Gatterdam.\\[1em] 

\begin{definition}[Rabin, Definition 7]
Let $G$ and $F$ be $\cE_n$-computable groups with $\cE_n$-admissible indices
$i_1$ and $i_2$, respectively. A homomorphism $\phi:G\maps F$ is
called $\cE_n${\em -computable} with respect to $i_1$ and $i_2$ if ....
(replace ``computable'' with ``$\cE_n$-computable'' and ``recursive'' by
``$\cE_n$-recursive'' throughout).
\end{definition}
\begin{theorem}[Rabin, Theorem 1]
Replace ``computable'' with ``$\cE_n$-computable'' and ``recursive'' by
``$\cE_n$-recursive'' throughout.
\end{theorem}
\begin{proof}
The proof goes through with the obvious modifications: one of which is
that the inverse function has to be explicitly defined. 
\end{proof}
\begin{proposition}[Rabin, Remark, Converse of Thm 1]
Same amendments: proof goes through.
\end{proposition}


Call the index defined in Theorem 1.2, by minimalisation, the {\em induced}
index on $G/H$. Then Theorem 1.2 says that the induced index is $\cE_n$-admissible if and only if $i(H)$ is $\cE_n$-decidable (recursive). This might
 be a good way to put things. 


\begin{definition}
Let $G$ be a group, $S$ a set disjoint from $G$ and $\pi:S\maps G$ a map. 
The ordered pair $(S,\pi)$ is called a {\em generating system} for
$G$ if the extension of $\pi$ to the free group $F(S)$ is surjective. 
We often abuse notation by refering to $S$ as a generating set and assuming
that $S$ is a subset of $G$. (See Cohen's book on combinatorial group theory,
for example.) 
\end{definition}

\begin{definition}[Standard Index]
Let $G$ be an $\cE_n$-computable group with generating set $(S,\pi)$ and 
let $F=F(S)$. Let $i_F$ be an $\cE_n$-index (or does it need to be $\cE_3$?)
and $i_G$ an 
 $\cE_n$-index for $G$. Then $G$ is  
said to be $\cE_n${\em -standard} with respect to $(S,\pi)$, 
$i_F$ and $i_G$ if 
the induced map $\bar\pi$ (as in defn 1.1 above)  is $\cE_n$-computable.
\end{definition}

It should be proved that this is equivalent to the C \& G definition, at 
least when the index for $F$ is $\cE_3$. This is in effect what Rabin 
does in his proof of his Theorem 1. He notes that, in the terminology
of his theorem, $\bar\phi=f$. 

\begin{theorem}[Rabin, Theorem 2]
Let $G$ be a finitely generated $\cE_n$-computable group .... Let $H$ be
a normal subgroup of $G$ such that $G/H$ is $\cE_n$-computabe with respect
to the induced index. Then $i(H)$ is $\cE_n$-decidable.  
\end{theorem}
\begin{proof}
\noindent{\bf Claim.} Let $G_1$ be an $\cE_n$-computable group with 
$\cE_n$-admissible index $i_1$ and let $\phi:G\maps G_1$ be a homomorphism
such that $\bar\phi$ is bounded by an $\cE_n$ function. Then 
$\bar\phi$ is $\cE_n$-computable. (I think this is the correct condition
on $\phi$ and is necessary.)

Once the above claim has been proved the theorem is proved by noting
that the induced index on $G/H$ is $\cE_n$-admissible and satisfies  
 $\bar\phi(j)\le j$, for all $j\in i(G)$.
\end{proof}

\begin{theorem}[Rabin, Theorem 3]
If $G$ is computable with $\cE_n$-index $i$ and $S$ is a subset 
of $G$ such that $i(S)$ is $\cE_n$-recursively enumerable then 
$G(S)$ is $\cE_n$-computable. If $i(S)$ is $cE_n$-recursive then
$G(S)$ is $\cE_n$-standard. 
\end{theorem}
\begin{proof}
My initial reading suggests the proof goes through, but there are 
a couple of things I'd need to check to be certain.
\end{proof}

\begin{corollary}[Rabin, Corollary]
Every finitely generated subgroup of an $\cE_n$-standard group
is $\cE_n$-standard. 
\end{corollary}
(Perhaps the condition f.g. is too strong here.)


\begin{lemma}[Rabin, Lemma 2 $+$ C \& G]
A finite or countably generated free group is $\cE_3$-computable.
\end{lemma}

An $\cE_n$-computable  group 
has $\cE_n${\em -solvable} word problem with respect to a generating system
$(S,\pi)$ if the set $i_F(\ker(\pi))$ is an $\cE_n$-decidable subset of $i_F(F(S))$, where $i_F$ is an $\cE_n$ index for $F=F(S)$. 

\begin{theorem}[Rabin Theorem 4, version 1]
Let $G$ be a group with generating system $(S,\pi)$. Then $G$ is 
$\cE_n$-standard with respect to $(S,\pi)$ if and only if 
$G$ has $\cE_n$-solvable word problem with respect to $(S,\pi)$. 
\end{theorem}
\begin{proof}
$>$: The map $\bar\pi$ is by definition $\cE_n$-computable, so from 
Proposition 1.3, $i_F(K)$ is $\cE_n$-decidable ($K=\ker(\pi)$). 

$<$: If $G$ has $\cE_n$-solvable word problem then from Theorem 1.2 and 
its proof  the induced index on $G$ is $\cE_n$-admissible: so $G$ 
is $\cE_n$-standard with respect to $(S,\pi)$.
\end{proof}
\begin{theorem}[Rabin 4, version 2]
If $G$ has $\cE_n$-solvable word problem with respect to finite system
of generators and $T$ is any finite generating system for $G$ then
$G$ has $\cE_n$-solvable word problem with respect to $T$ (and is
$\cE_n$ standard w.r.t. $T$).
\end{theorem}
Proved using Theorem above and the Corollary above and some arguement
showing that a $\cE_n$ index for $T$ can be built from the given 
$\cE_n$ index. 
\section{Pregroups}\label{sec:pregroups}

Let $(P,D)$ be a pregroup with partial binary operation 
$m:D\maps P$. Suppose that 
\begin{itemize}
\item 
$i:P\maps \NN$ is an injective function and 
\item setting $D^\prime=
\{(x,y)\in \NN\times \NN\,:\, x=i(p), y=i(q), (p,q)\in D\}$  
a function 
$m^\prime:D^\prime \maps \NN$ is defined  by $m^\prime(i(p),i(q))=i(m(p,q))$, 
for 
all $(p,q)\in D$ 
and 
\item $j$ is the function given
by $j(i(p))=i(p^{-1})$, for all $p\in P$.
\end{itemize}
Then $(P,D)$ is said to have 
{\em index} $(i,m^\prime,j)$. 
\begin{definition}
A pregroup $(P,D)$ is $\cE_n(A)$-{\em computable} with respect to
index $(i,m^\prime,j)$ if 
\be
\item $i(P)$ is $\cE_n(A)$-decidable;
\item $D^\prime$ is $\cE_n(A)$-decidable;
\item $m^\prime$ is $\cE_n(A)$-computable and 
\item $j$ is $\cE_n(A)$-computable.
\ee
\end{definition}
\begin{comment}
Write $(p_1,\ldots ,p_n)\in P$ if, with some bracketing, $p_1\cdots p_n\in P$.
Let $K=\{(p_1,\ldots ,p_n)\in P\,:\, n\ge 1, p_1\cdots p_n=1\}$ and 
let $K^\prime=\{(i(p_1),\ldots ,i(p_n))\,:\,   (p_1,\ldots ,p_n)\in K\}$. 
\begin{definition}
A pregroup $(P,D)$ is $\cE_n(A)$-{\em standard} with respect to
index $(i,m^\prime,j)$ if $K^\prime$ is $\cE_n(A)$-decidable. 
\end{definition}

A subset $X$ of $P$ is a {\em generating set} for $P$ if, for all $p\in P$, 
there exists a sequence $(p_1,\ldots p_n)\in P$ such that 
$p_i\in X\cup X^{-1}$  and, with some choice of bracketing, 
$p_1\cdots p_n=p$.
\end{comment}
 
\begin{theorem}
Let $G=\cU(P)$, the universal group of the pregroup $(P,D)$. 
If $(P,D)$ %has finite generating set $X$ and is standard with respect
is $\cE_n(A)$-computable with respect 
to the index $(i,m^\prime,j)$ then $G$ is $\cE_{n}(A)$-standard.
\end{theorem}
\begin{proof}
Let $F$ be the free group on $P$ and let $(i_F, m_F, j_F)$ be 
a $\cE_3$ index for $F$. The index $i$ extends to an $\cE_n$-index
$i_1$ on reduced words $(p_1,\ldots, p_k)$ in $P^*$ (that is $(p_i,p_{i+1})\notin
D$) in the usual way. 
Now we use minimisation to determine representatives of 
equivalence classes of reduced words over $P^*$. 
Define 
\[E(x,y)\iff x\in i_1(P^*)\wedge y\in i_1(P^*)\wedge m^\prime(j(x),y)=i(1_P).\]
Next 
\[r(j)=\min\{k\le j\wedge k\in i_1(P^*) \wedge E(k,j)\}.\]
Then $(ri_1,rm^\prime,rj)$ is an $\cE_n$-index for $G$. Furthermore 
the induced map from $F(P)$ to $G$ is $i_1i^{-1}$ so is $\cE_n$-computable.
\end{proof}

Next discuss applications to the word problem and the level of rewriting
systems on the hierarchy. There is some literature on rewriting systems
and the Grzgorczyk hierarchy that needs checking. 
\section{Pregroups the conjugacy problem and the G hierarchy}
We shall take the lead from a short paper that has not yet been quite completed: but should be done in a couple of weeks. 
%%%%%%%%%%%%%%%%%%%%%%
\bibliographystyle{abbrv}
\bibliography{traces}
\end{document}
